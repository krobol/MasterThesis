\documentclass[thesis.tex]{subfiles}
\begin{document}

\chapter{Introduction}
\label{chap:introduction}

%\begin{figure}[h]
%\centering
%\includegraphics[width=\textwidth]{giphenomenons}
%\caption{\cite{bib:RealtimeGIOverview} Photography of a scene with various global illumination effects: Diffuse and specular bounces, caustics and scattering.}
%\label{fig:giphenomenons}
%\end{figure}

\section{Medical Background}


\section{Problem Definition}

\section{Main Contribution}

%\begin{itemize}
%\item We introduce a new view-space based light cache selection in cascaded radiance volumes. Our approach is able to limit the lighting computations to the set of actually needed caches and keeps %the overall memory consumption low, as we do not need to save any complex data in a grid. Additionally, we provide implementation details on how to speed up this process using shared memory.
%\item We present dynamic, scalable and comparatively accurate indirect shadowing for light caches using voxel cone tracing on groups of virtual lights.
%\item We propose the new idea of hemispherical specular environment maps for glossy reflections as an addition to classic spherical harmonic based radiance caches.
%\end{itemize}

\subfilebib % Makes bibliography available when compiling as subfile
\end{document}
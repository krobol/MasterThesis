\documentclass[thesis.tex]{subfiles}
\begin{document}
\chapter{Introduction}
\label{chap:introduction}
Aortic dissection (AD) describes a condition of the aortic vessel wall in which a tear in the intima, the innermost layer of the aortic vessel wall, allows blood to flow between the layers of the vessel wall. This causes the layers of the aortic vessel wall to be forced apart by the blood pressure, drastically increasing the diameter of the aorta and forming a secondary flow channel, the false lumen (FL). If untreated the FL keeps growing in size and propagates along the actual flow channel, the true lumen (TL).

According to large population-based studies the incidence of AD steadily increased since 1987 \cite{olsson2007thoracic}. The measured prevalence ranges from 6 cases / 100.000 to 16.3 cases / 100.000 per year \cite{olsson2007thoracic,goldfinger2014thoracic}. 

Patients that survive an aortic dissection and leave the hospital are recommended to undergo medical therapy and have their aorta reassessed regulary by non-invasive imaging methods \cite{olsson2007thoracic,desanctis1987aortic,baliga2014role}. The resulting imaging studies from the follow-up have been reported to improve diagnosis, prediction and management of aortic dissections \cite{doi:10.1161/CIRCULATIONAHA.117.031264} and generally help to improve our understanding of the aortic function \cite{baliga2014role}. As the importance of the follow-up period increases with the increasing incidence of aortic dissections, new models to represent aortic dissections should be considered. 


\section{Problem Definition} \label{problem_def}

The main goal of this thesis is to find a representation of ADs. Ideally, this representation should allow us to store the shape of the AD in a compact manner, while maintaining high accuracy. It should also be able to easily utilize the representation of the AD for 3D surface visualization and for tasks that might be done with vascular structures or ADs, such as taking measurements, comparing ADs or predicting the development of the AD. This means that the representation has to be easily interpretable.

Current 3D surface visualization techniques of vascular structures can be roughly classified as model-based approaches or model-free approaches. \cite{preim20083d} gives an overview of such techniques. Model-based approaches explicitly construct the vascular structure using geometric primitives. If we approximate an AD using a model-based approach, we receive a compact representation of the AD that is easy to interpret and therefore can be used for most task. Unfortunaly, the accuracy of model-based methods is low because they usually assume that vessels have a circular or elliptical shape.  

Model-free methods instead use the underlying data to create surfaces. They are more accurate than model-based approaches and can, depending on the technique used, produce smooth surfaces that are pleasing to the human eye and do not contain distracting artifacts. However, the underlying data of model-free approaches is relatively hard to interpret and not suited for most other tasks. 

\section{Main Contribution}

Our main contribution is the investgation of Elliptic Fourier Descriptors (EFD) for the purpose of capturing the shape of ADs. We will try to use EFDs to describe the contours of the lumina (TL and FL), which we extract from cross-sections. Therefore, a segmentation of the true and false lumen is necessary including both of their centerlines. The extracted contours have to be described by chain codes before they can be used to compute the EFDs. We will investigate the differences that occur when different chain codes are used and will try to give a recommendation of which chain code should be used. We will also use the EFDs with 3D surface visulization techniques to render and visualize the AD. Moreover, we will investigate the accuracy of the EFDs.

\chapter{Medical Background}

%\todo[inline]{TODO: Add image of the aorta}
%\todo[inline]{TODO: Add info about landmarks}
%\todo[inline]{TODO: differences to hematom etc.}
Aortic dissection (AD) describes a condition of the aortic vessel wall in which a tear in the intima, the innermost layer of the aortic vessel wall, allows blood to flow between the layers of the vessel wall. This causes the layers of the aortic vessel wall to be forced apart by the blood pressure, drastically increasing the diameter of the aorta and forming a secondary flow channel, the false lumen (FL). If untreated the FL keeps growing in size and propagates along the actual flow channel, the true lumen (TL).

The blood flowing in the FL can re-enter the TL, but often causes uncontrollable blood pressure. The expanded FL has a high risk of rupture and can physically compress \cite{criado2011aortic} and damage the aorta or other vessels connected to it, causing insufficient blood supply and organ failure \cite{meszaros2000epidemiology,desanctis1987aortic}. 

\section{Anatomy of the Aorta}

The aorta is a large artery connected to the left ventricle of the heart. It can be divided into two parts, the thoracic and abdominal aorta. The thoracic aorta is the part of the aorta that is connected to the heart, then forms an arch and continues downward. At the level of the diaphragm the aorta is called abdominal aorta. The thoracic aorta can be divided into three parts: the ascending aorta, the aortic arch, and the descending aorta.

\textbf{Ascending Aorta}

The ascending aorta can be further divided into two parts: the aortic root, and the upper portion of the ascending aorta. 
The aortic root starts at the aortic annulus, which is a ring-like structure at the beginning of the aortic valve, and extends up to the sinotubular junction. The annulus and aortic sinuses belong to the aortic root. The upper portion of the ascending aorta starts at the sinotubular junction and extends up to the aortic arch.

\textbf{Aortic Arch}

The aortic arch starts after the ascending aorta at the innominate artery and extends up to the left subclavian artery. In total three important arteries branch of the aortic arch: the innominate artery, the left common carotid artery and the left subclavian artery. 

\textbf{Descending Aorta}

The descending aorta begins at the  left subclavian artery of the aortic arch and continues downwards up to the level of the diaphragm. At each level of the spine a pair of intercostal arteries branches of the descending aorta.

\textbf{Abdominal Aorta}

The abdominal aorta is the direct continuation of the descending aorta. Important arteries that branch of the abdominal aorta include the renal arteries, which supply the kidneys with blood.  

\section{Classification of Aortic Dissections}

%\todo[inline]{TODO: add distribution of types}
\textbf{DeBakey Classifcation}

The DeBakey classification \cite{desanctis1987aortic,goldfinger2014thoracic,criado2011aortic} classifies dissections by the location of the initial tear in the intima and the extent of the FL. It distinguishes three classes: type \uproman{1}, type \uproman{2}, and type \uproman{3} dissections. 

A dissection belongs to type \uproman{1} or type \uproman{2} if the initial tear is located in the ascending aorta. Type \uproman{3} dissections have their tear located in the descending thoracic aorta. The difference of type \uproman{1} and type \uproman{2} dissections is that in type \uproman{1} dissections the FL extends beyond the ascending aorta while type \uproman{2} dissections are confined to the ascending aorta. Type \uproman{3} dissections can be further divided into two classes: type \uproman{3} A and type \uproman{3} B dissections. Type \uproman{3} A dissections do not extend below the diaphragm, while Type \uproman{3} B dissection do.

\textbf{Stanford Classification}

Another commonly used classification is the Stanford classification \cite{desanctis1987aortic,goldfinger2014thoracic,criado2011aortic} which classifies the dissections according to the involvement of the ascending aorta. There are two classes: Type A dissections that involve the ascending aorta, and Type B dissections that do not. 

\textbf{Temporal Classification}

The mortality of patients with AD is greatly increased in the first two weeks after onset of symptoms. To reflect these changes of mortality over time the following temporal classification of ADs is commonly used: When a patient with an AD survives at least two weeks, the AD is classified as chronic. Otherwise the AD is classified as acute \cite{olsson2007thoracic,desanctis1987aortic,criado2011aortic}.

Recently, a new temporal classifcation was proposed by the IRAD \cite{doi:10.1161/CIRCULATIONAHA.117.031264}. The new temporal classification differentiates between 4 time periods. ADs are classifed as hyperacute in the first 24 hours after symptom onset, as acute from the 2 to 7 day, as subacute from the 8 to 30 day, and as chronic after 30 days.  

\section{Diagnosis}

%\todo[inline]{TODO: describe the data we use}
The mortality of AD has been estimated to be 1\% to 2\% per hour immediately after symptom onset \cite{doi:10.1161/CIRCULATIONAHA.117.031264,meszaros2000epidemiology,goldfinger2014thoracic}. This means a fast diagnosis is of particular importance. In 2010 a guideline was created \cite{hiratzka20102010} to ensure a fast and accurate diagnosis. The guideline uses risk factors that include symptoms, medical history, family history of aortic disease, and findings of examination by imaging modalities to estimate the risk level of patients. The guideline also describes how the different imaging modalities can be used to confirm or exclude an AD and which imaging modalities may delay the diagnosis or are only suited for specific situations or types of ADs. In this section we give an overview of available imaging modalities that are usually used in the context of ADs.

\textbf{Echocardiography}

Echocardiography is an imaging modality that uses ultrasound to create live images of the heart and aorta. In the IRAD 25\% of the inital diagnoses were made using echocardiography making it the second most used imaging modality for initial diagnosis of AD \cite{doi:10.1161/CIRCULATIONAHA.117.031264}. The main advantages of using echocardiography is that it is readily available and can be perfomed quickly, even at the bedside \cite{baliga2014role}. There are several types of echocardiography with different uses in the context of AD.

\textbf{Transthoracic Echocardiography}

Transthoracic echocardiography (TTE) is a non-invasive type of echocardiography that is performed by placing an ultrasound transducer on the chest wall. It can be used to visualize the ascending aorta to just above the sinotubular junction and part of the descending aorta \cite{baliga2014role}. This means it is not suited to visualize the upper portion of the ascending aorta and the aortic arch \cite{baliga2014role}. For Stanford type A dissections it has a sensitivity of 78\% to 100\% and for Stanford type B dissection the sensitivity is 31\% to 55\% \cite{baliga2014role}. The overall sensitivity for ADs is only at 59\% to 83\% with a specificity of 63\% to 93\%. The low sensitivity means that it cannot be used to exclude an AD and therefore should not be used for initial diagnosis. Despite these disadvantages and the low overall sensitivity, according to \cite{baliga2014role} TTE is still useful for rapid assement of complications in the case of emergency. 

\textbf{Transesophageal Echocardiography}

Transesophageal echocardiography (TEE) is an invasive type of echocardiography that is performed by placing a probe that contains an ultrasonic transducer in the esophagus of the patient. TEE can be used to visualize the ascending aorta, parts of the arch and the descending aorta with high spatial resolution \cite{baliga2014role}. TEE also has a high sensitivity of up to 99\% and a specificity of 89\% \cite{baliga2014role}. However, the upper portion of the ascending aorta and the branches of the aortic arch \cite{shiga2006diagnostic,baliga2014role}, as well as the branches of the abdominal aorta cannot be adequately assested by TEE \cite{baliga2014role}. Another disadvantage of TEE is that patients have to be sedated. The resulting images also often contain artifacts that can be mistaken for a FL. Therefore, an experienced operator is needed for image aquisition and interpretation \cite{shiga2006diagnostic,baliga2014role}. These disadvantages make TEE not the first choice, but it is relatively fast compared to CT and MRI, has a high sensitivity and specifcity, and therefore can be useful in emergency situations where fast assement of the aorta is necessary \cite{shiga2006diagnostic,baliga2014role}. TEE is also used to perform endovascular stent grafting and for confirming success after surgery \cite{baliga2014role}. 

%\textbf{Computer Tomography} 
%\todo[inline]{TODO}

%\textbf{Magnetic Resonance imaging}
%\todo[inline]{TODO}
\section{Therapy} 

Depending on the risk level that was determined in the diagnosis, the type of the AD, and the state of the patient different therapies are recommened by the guideline \cite{hiratzka20102010}. The recommendations are visualized in a flow chart and are easily understood.

The guideline usually recommends surgery or interventional management for patients with Stanford type A dissection. If the patient's condition does not permit surgery, medical treatment is recommended. For Stanford type B dissections medical management is usually recommended. If complications occur, either oral medication, or surgery / interventional management is recommended depending on the specific complication. 

Before discharge, the guideline recommends to start a follow-up imaging study, then reasses the aorta at 1, 3, 6, and 12 months after discharge and annually thereafter \cite{hiratzka20102010}.

The recommendations in the guideline greatly overlap with the management of patients by the IRAD since 1999 \cite{doi:10.1161/CIRCULATIONAHA.117.031264}. According the the IRAD, 86\% of Stanford type A dissections were managed surgically in the last 20 years. Of Stanford type B dissections, 57\% were treated medically in 2013, 31\% were treated by endovascular management, and 8\% were treated surgically.  


%\begin{itemize}
%\item We introduce a new view-space based light cache selection in cascaded radiance volumes. Our approach is able to limit the lighting computations to the set of actually needed caches and keeps %the overall memory consumption low, as we do not need to save any complex data in a grid. Additionally, we provide implementation details on how to speed up this process using shared memory.
%\item We present dynamic, scalable and comparatively accurate indirect shadowing for light caches using voxel cone tracing on groups of virtual lights.
%\item We propose the new idea of hemispherical specular environment maps for glossy reflections as an addition to classic spherical harmonic based radiance caches.
%\end{itemize}

\subfilebib % Makes bibliography available when compiling as subfile
\end{document}
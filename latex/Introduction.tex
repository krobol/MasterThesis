\documentclass[thesis.tex]{subfiles}
\begin{document}
\chapter{Introduction}
\label{chap:introduction}

%\begin{figure}[h]
%\centering
%\includegraphics[width=\textwidth]{giphenomenons}
%\caption{\cite{bib:RealtimeGIOverview} Photography of a scene with various global illumination effects: Diffuse and specular bounces, caustics and scattering.}
%\label{fig:giphenomenons}
%\end{figure}

According to large population-based studies the incidence of Aortic Dissections steadily increased since 1987 reaching 16.3/100 000 per year for men and 9.1/100 000 per year for women in 2002. At the same time the mortality drastically decreased. Some of the reasons for this development are improved diagnostics, better management of patients, a more active attitude towards surgery and improved non-invasive imaging techniques. \\
The 30-day mortality of patients that did not underwent surgery decreased from 42\% in 1987 to 26\% in 2002. For patients that underwent surgery it decreased from 25\% to 13\% in 2002. Patients that survive the aortic dissection and leave the hospital are recommended to undergo medical therapy and have their aorta reassessed annualy by non-invasive imagine techniques. \\ However even with annually reassesments, the 10-year actuarial survival rate was measured to only range from 30\% to 60\%. It is possible this long-term survival rate could still be increased by improving the models computers use predict the development of the aorta.
\\


  
\section{Medical Background}
Aortic Dissection (AD) describes a condition of the aortic vessel wall in which a tear in the intima, the innermost layer of the aortic vessel wall, allows blood to flow between the layers of the vessel wall. This causes the layers of the aortic vessel wall to be forced apart by the blood pressure, drastically increasing the diameter of the aorta and forming a secondary flow channel, the false lumen (FL). If untreated the FL keeps growing in size and propagates along the actual flow channel, the true lumen (TL). \\
The blood flowing in the FL can re-enter the TL, but often causes uncontrollable blood pressure. The expanded FL has a high risk of rupture and can physically compress\cite{criado2011aortic} and damage the aorta or other vessels connected to it causing insufficient blood supply and organ failure \cite{meszaros2000epidemiology,desanctis1987aortic}. 

\subsection{Classification of Aortic Dissections}
The DeBakey classification \cite{desanctis1987aortic,goldfinger2014thoracic,criado2011aortic} can be used to to classify dissections by the location of the initial tear in the intima and the extent of the FL. It distinguishes four classes: Type \uproman{1}, Type \uproman{2}, Type \uproman{3} A, and Type \uproman{3} B dissections. \\ A dissection belongs to Type \uproman{1} or Type \uproman{2} if the initial tear is located in the ascending aorta while for Type \uproman{3} dissections the tear has to be located in the descending thoracic aorta. The difference of Type \uproman{1} and Type \uproman{2} dissections is that in Type \uproman{1} dissections the FL extends beyond the ascending aorta while Type \uproman{2} dissections are confined to the ascending aorta. Type \uproman{3} A dissections do not extent below the diaphragm, while Type III B dissection do. 
\\ \\
Another simpler classification is the Standford classification \cite{desanctis1987aortic,goldfinger2014thoracic,criado2011aortic} that classifies the dissections according to the involvement of the ascending aorta. There are two classes: Type A dissections that involve the ascending aorta, and Type B dissections that do not. 
\\ \\
There also is a temporal classification of ADs. When a patient with an AD survives at least two weeks, the AD is classified as chronic. Otherwise the AD is classified as acute.\cite{olsson2007thoracic,desanctis1987aortic,criado2011aortic}

\subsection{Diagnosis}

\section{Problem Definition}

\section{Main Contribution}

%\begin{itemize}
%\item We introduce a new view-space based light cache selection in cascaded radiance volumes. Our approach is able to limit the lighting computations to the set of actually needed caches and keeps %the overall memory consumption low, as we do not need to save any complex data in a grid. Additionally, we provide implementation details on how to speed up this process using shared memory.
%\item We present dynamic, scalable and comparatively accurate indirect shadowing for light caches using voxel cone tracing on groups of virtual lights.
%\item We propose the new idea of hemispherical specular environment maps for glossy reflections as an addition to classic spherical harmonic based radiance caches.
%\end{itemize}

\subfilebib % Makes bibliography available when compiling as subfile
\end{document}
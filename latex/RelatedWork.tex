\documentclass[thesis.tex]{subfiles}
\begin{document}

\chapter{Related Work}
\section{Fourier Transform}

Was genau ist die Fourier transformation ? \\
todo: time domainund frequency domain erklären \\
todo: figures \\ \\

In mathematics the fourier transform is a function that approximates a signal in the time domain by a sum of sine and cosine functions with different frequencies. The result is then in the frequency domain. This means the fourier transform finds the frequencies, amplitudes and phases that are necessary to best approximate the signal in the time domain.
\\ 
It is possible to visualize how the fourier transform works: Assume we want to approximate a given sine function by going along the outline of a circle over time. For each timestep, which can be infinitely small, we use the time as the x-values of our apprimation. For the y-values of the approximation we use the y-value of the position on the circle outline at each time. There are three parameters that can be adjusted in this process: the radius of the circle, the speed at which we go along the outline, and the starting position on the outline. By changing the radius of the circle we change the amplitude of the approximation, by changing the speed we change the frequency of the approximation and by changing the starting position on the outline we change the phase of the approximation. To find the optimal values to actually approximate the given sine wave, the fourier transform solves the given signal for this three parameters.
\\ So far we only approximated a simple sine wave. To approximate a more complex signal this process can be recursively repeated: For each circle we can place the center of the circle on the outline of the previous circle, if there is one. This enables us to perfectly reconstruct most functions with infinite frequencies. 
\\



Formel und erklärung was sie macht \\
Visualisierung \\
Limits und probleme \\
evntuell fehelrberechnung usw

\section{Elliptic Fourier Descriptors}
Wieso sind es 4 coefficients ? \\
Visualisierung der coefficienten ? \\
DC Components ?

\section{Chain Codes}
Wieso brauchen wir chain codes ? \\
Was sind chain codes ? \\
Welche chain codes gibt es? 



\subfilebib % Makes bibliography available when compiling as subfile
\end{document}
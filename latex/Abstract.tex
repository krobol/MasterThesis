\documentclass[thesis.tex]{subfiles}
\begin{document}

\addchap{Abstract}

In this thesis we investigate the utility of Elliptic Fourier Descriptors (EFD) to model the shape of Aortic Dissections. We use the EFDs to demonstrate how the shape of Aortic Dissections can be reconstucted and used for visualization and rendering. We also propose a way to prevent overlapping of the reconstructed lumen and investigate the effects of using different chain codes for the computation of the EFDs.

The resulting renderings depict the Aortic Disections more accurately than current state of the art methods, that approximate vascular structures by ellipses. Additionally, we determine that the Vertex Chain Code is better suited for this purpose than the Freeman chain code, as the reconstruction of the Aortic Dissection shape needs less harmonics to achieve the same results as with the Freeman chain code.

\addchap{Kurzfassung}

In dieser Masterarbeit untersuchen wir, ob sich Elliptic Fourier Descriptors (EFD) dazu eignen, die Form von Aortic Dissections zu modelieren. Wir verwenden EFDs, um zu demonstrieren, wie die Form der Aortic Dissections näherungsweise wiederhergestellt werden kann und wie das Ergebnis dann für die Visualisierung und Rendering der Aortic Dissection verwendet werden kann. Außerdem, zeigen wir, wie Überlappungen der wiederhergestellten Blutgefäße der Aorta beim Rendering verhindert werden können. Zusätzlich untersuchen wir die Auswirkungen von verschiedenen Chain Codes, die für die Berechnung der EFDs verwendet werden, auf das Ergebnis. 

Unsere Rendering Ergebnisse bilden die Aortic Dissections, die uns in Form von MRT Daten vorliegen, genauer ab als bisherige Methoden, die Blutgefäße durch Ellipsen annähern. Eines unserer Ergebnisse ist auch die Erkenntnis, dass der Vertex Chain Code sich für diese Aufgabe besser eigenet als der Freeman Chain Code, da er weniger Harmonics benötigt, um gleichwertige Ergebnisse zu liefern wie mit dem Freeman Chain Code.

\end{document}